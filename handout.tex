% -*- japanese-LaTeX -*-
% 
% 
% Copyright (c) 2022, Hiroyuki Ohsaki.
% All rights reserved.
% 
% $Id: paper.tex,v 1.3 2022/06/28 02:09:15 ohsaki Exp $
% 
\documentclass[twocolumn,a4paper, dvipdfmx]{ieicejsp}
\usepackage{amsmath}
\usepackage{cite}
\usepackage{insertfig}
\usepackage{rangecite}
\usepackage{times}
\usepackage{url}
\usepackage{revhistory}

\renewcommand{\baselinestretch}{0.01}

\makeatletter
\makeatother

\title{{\bf 未知のグラフにおけるレジリエンス向上のための \\アンカーノード探索手法に関する一検討}\\
  {\normalsize A Study on Identifying Anchor Nodes for Maximizing Resilience on Unknown Graphs}
}


\author{
  37022497 石本 哲郎
%%   井上 桃 $^2$ \\ Momo Inoue \and
%  井上 翔太 $^2$ \\ Shota Inoue \and
%  大崎 博之 $^2$ \\ Hiroyuki Ohsaki
}

\affliate{%    
  関西学院大学 工学部 情報工学課程 \\
  2025年度 卒業研究 審査資料  
  }


\begin{document}

\maketitle


\section{はじめに}

現代の高度情報化社会において、通信ネットワークや社会基盤システムは不可
欠な存在であり、そのロバスト性、すなわちレジリエンスを確保することは極めて
重要な課題である。

ネットワークレジリエンスを向上させるための一つのアプローチとして、
ネットワークの機能維持のために固定するアンカーノードの戦略的配置が提案されている~\cite{Teng24:WWW}。
% 。Teng らの研究~\cite{Teng24:WWW} で、この問題をフォロワー最大化問題として定式化し、
% ネットワークのレジリエンスを定量化する枠組みが提示された。同研究では、限られたコ
% ストの下でネットワークのレジリエンス向上に最も寄与するノードを逐次的にアンカー
% として選択する貪欲アルゴリズム AdvGreedy を提案しており、その有効性
% が示されている。
Tengらの研究では、ネットワークのレジリエンス向上を目的としたアンカーノードの戦略的配置をフォロワー最大化問題として定式化し、
限られたコストでレジリエンス向上に最も寄与するノードを逐次的に選択する貪欲アルゴリズムAdvGreedyの有効性を示している。

% しかしながら、既存研究では、ネットワーク全体のトポロジカルな構造が完全
% に既知であり、任意のノードの状態を即座に把握できるという仮定に基づいて
% いる。

% そこで本研究では、単一のエージェントがランダムウォークによって未知のネッ
% トワークを局所的に探索し、その過程で得られる部分的な情報のみを用いてア
% ンカーノードを選定する、というシナリオを想定し、不完全な情報下における
% アンカー選定戦略の有効性を明らかにすることを目的とする。-> geminiで修正
既存研究ではネットワーク全体のトポロジとノードの状態が既知であることを前提としている。
本研究では、単一エージェントがランダムウォークにより未知のネットワークを局所的に探索し、
その部分的な情報のみを用いてアンカーノードを選定するシナリオを想定し、不完全情報下における
アンカー選定戦略の有効性を明らかにすることを目的とする。


\section{フォロワー最大化問題}

フォロワー最大化問題は、ネットワークのレジリエンスを定量的に評価し、そ
の向上策を最適化するための理論的枠組みである~\cite{Teng24:WWW}。

本問題では、ネットワークを構成するノードを「アンカー」と「フォロワー」
の2種類に分類している。アンカーとは、障害の影響を受けない特別なノード
群を指す。一方、フォロワーとは、アンカー以外のすべての一般ノードであり、
これらは潜在的な障害のリスクに晒されている。あるフォロワーノードが保護
されている状態とは、そのノードから少なくとも一つのアンカーノードに対し
て、障害のない安定した経路上で到達可能であることを意味する。

本問題の目的は、限られた予算内で保護されるフォロワーの総数を最大化するア
ンカーノード集合を発見することにある。形式的には、グラフ $G = (V, E)$
とアンカー設置の総予算 $b$ が与えられたとき、$|A| \leq b$ を満たすアン
カー集合 $A \in V$ のうち、保護されるフォロワーノードの総数、すなわち
レジリエンス利得を最大化するものを求める組合せ最適化問題として定式化さ
れる。%% この問題は NP 困難であることが知られており、実用的な時間内に厳密
%% 解を求めることは極めて困難である。


\section{ランダムウォークによるアンカーノード探索}

% 本研究では、ネットワーク全体のトポロジが未知であるという制約下で、その
% 構造情報を獲得する現実的な手段としてランダムウォークを用いる。ランダム
% ウォークは、探索エージェントが現在位置するノードの近傍情報のみに基づき、
% 次に遷移するノードを確率的に選択するプロセスである。エージェントが訪問
% したノードとエッジの集合は、部分グラフとして記録される。

% アンカーノードの選定は、ランダムウォークによって得られた部分グラフ上で
% 行われる。具体的には、探索エージェントが所定のステップ数 $k$ の探索を
% 終えた時点で、その軌跡から観測されたノード集合およびエッジ集合によって構成さ
% れる部分グラフ $G^*_k = (V^*_k, E^*_k)$ を構築する。そして、この
% $G^*_k$ を対象として、前述のフォロワー最大化問題を解くために AdvGreedy
% アルゴリズム~\cite{Teng24:WWW} を適用する。これにより、不完全な情報の
% みに基づいたアンカーノードの候補集合 $A_k$ が得られる。

本研究では、未知のネットワークにおいて、その構造情報を獲得する現実的な手段として
ランダムウォークを用いる。ランダムウォークは、エージェントが現在位置するノードの
近傍情報に基づき、次に遷移するノードを確率的に選択するプロセスであり、
訪問したノードとエッジは部分グラフとして記録される。

アンカーノードの選定は、
ランダムウォークで得られた部分グラフ上で行う。具体的には、探索エージェントがステップ数 k の探索後、
その軌跡から構築された部分グラフ$G^*_k = (V^*_k, E^*_k)$に、
フォロワー最大化問題を解くAdvGreedy アルゴリズム~\cite{Teng24:WWW} を適用する。
これにより、不完全な情報に基づくアンカー候補集合 $A_k$を得る。


\section{実験}

\insertfigs{random-result}{ランダムグラフでのレジリエンス利
  得}{ba-result}{BA グラフでのレジリエンス利得}

%% 提案手法の有効性を多角的に検証するため、
ランダムウォークによるアンカーノード探索の有効性を検証するため、構造的
特徴が異なる2種類のグラフモデル (ランダムグラフ、BA : Barab\'{a}si--Albert
グラフ) を用い、それぞれについてグラフを一つずつ生成してシミュレーションを実施した。
それぞれのグラフは、ノード数を $100$、平均次数を約$4.5$ として生成し、アンカー選定の予算
$b$ は全ノード数の $5\%$、すなわち $5$ とした。

%% それぞれのグラフに対して、10 個のランダムな始点ノードにて、単純ランダ
%% ムウォーク、非後退ランダムウォーク、近傍回避ランダムウォーク、自己回避
%% ランダムウォークを 10 回実行することによって得た部分グラフからアンカー
%% ノード集合 $A_k$ を決定し、各ランダムウォークのステップ $k$ に対するレ
%% ジリエンス利得~\cite{Teng24:WWW} を評価した。

%% ランダムグラフと BA グラフでの各ランダムウォークの探索ステップ $k =
%% 50, 100, ... 300$ の $A_k$ に対するレジリエンス利得の図 (図 1 および
%% 2) から、ランダムウォークのグラフを広く探索できる能力が、重要なアンカー
%% ノードの獲得にも有効であることががわかる。元のグラフを AdvGreedy によっ
%% て解いた際のレジリエンス利得は 15 と 30 であるため、図 1 と 2 の両方の
%% 結果で、SARW が速い段階でレジリエンスを最大化するようなアンカーを決定
%% できていることがわかる。また、ほぼ同等の結果である NBRW と VARW につい
%% で、SRW の結果が最もレジリエンス利得を得られない結果となり、これらの結
%% 果から、元のグラフの頂点を広く探索できるランダムウォーク手法ほど、早期
%% に高いレジリエンス利得を得ることが予想される。

本実験では、アンカー選定の前提となる探索戦略として、単純ランダムウォー
ク (SRW)、非後退ランダムウォーク (NBRW)、近傍回避ランダムウォーク
(VARW)、そして自己回避ランダムウォーク (SARW) を用意し、それぞれのグラ
フに対し、10個の異なる始点から各ランダムウォークを 10 回ずつ実行し、探
索ステップ数 $k = 50, 100, 150 ... 300$ ごとに得られた部分グラフからア
ンカー集合 $A_k$ を決定した。決定したアンカーノードを元グラフに適用したときに得られる
レジリエンス利得の平均値と95\%信頼区間を計測した。信頼区間は極めて小さく、プロットから識別できなかったため、図示を省略した。

% >>
% 評価指標として、得られたアンカーノード集合 $A_k$ を元のグラフに
% 適応した際のレジリエンス利得~\cite{Teng24:WWW} を計測した。
% その結果、ランダムグラフとBAグラフの評価指標はそれぞれ15,30となった。

評価指標として元のグラフで得られるレジリエンス利得を計測した。
元のグラフに対してAdvGreedyを用いて得たアンカーノードによるレジリエンス利得は、
ランダムグラフで15、BAグラフで30である。
% >>

% 図 1 および図 2 に示す実験結果は、いずれのグラフモデルにおいても、より
% 広い範囲を効率的に探索できるランダムウォーク戦略ほど、早期に高いレジリ
% エンス利得を達成する傾向を示している。特に、一度訪問したノードを積極的
% に避ける SARW が最も優れた性能を示し、探索ステップの初期段階で最適解に
% 近いレジリエンス利得を達成した。これに対し、SRW は、同一領域を繰り返し
% 探索する傾向が強く、性能の向上が最も緩やかであった。
% geminiの指摘箇所(2)を反映したものを以下に書く

図 1 および図 2 に示す実験結果から、本研究で設定した条件下においては、
いずれのグラフモデルにおいても、より広い範囲を効率的に探索できるランダム
ウォーク戦略ほど、早期に高いレジリエンス利得を達成する傾向が見られた。
% 特に、一度訪問したノードを積極的に避ける SARW が最も優れた性能を示し、
% 探索ステップの初期段階で最適解に近いレジリエンス利得を達成した。これに
% 対し、SRW は同一領域を繰り返し探索する傾向が強く、性能の向上が最も緩やである
% という結果が得られた。<< 削った

\section*{謝辞}
本研究の一部は JSPS 科研費 24K02936 の助成を受け
たものである

\renewcommand{\em}{\it}
\bibliographystyle{ieeetr}
\bibliography{bib/tetsuro}

\end{document}
